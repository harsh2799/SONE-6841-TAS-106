% This is samplepaper.tex, a sample chapter demonstrating the
% LLNCS macro package for Springer Computer Science proceedings;
% Version 2.20 of 2017/10/04
%
\documentclass[runningheads]{llncs}
\usepackage{amsmath}
\usepackage{hyperref}
%
\usepackage{graphicx}
% Used for displaying a sample figure. If possible, figure files should
% be included in EPS format.
%
% If you use the hyperref package, please uncomment the following line
% to display URLs in blue roman font according to Springer's eBook style:
% \renewcommand\UrlFont{\color{blue}\rmfamily}

\begin{document}
%
\title{Get Users Involved As Early As Possible}
%
%\titlerunning{Abbreviated paper title}
% If the paper title is too long for the running head, you can set
% an abbreviated paper title here
%
\author{Harsh Mishra | Student ID : 40206570}
%

%
\institute{Computer Science and Software Engineering , Concordia University \\
\vspace{10pt}  VCS - Github: \href{https://github.com/harsh2799/SONE-6841-TAS-106}{https://github.com/harsh2799/SONE-6841-TAS-106}}

%
%
\maketitle

\tableofcontents

\newpage

\section{Abstract}


\section{Introduction}
The software development industry is always changing, and a more user-centric and collaborative approach has replaced the old methods of working on projects in secret. In the past, software projects were developed in isolation, with engineers secretly gathering user needs and disclosing the finished product only after it was finished. However, this approach frequently resulted in a glaring mismatch between what users expected from the software and what was actually provided, which left them unhappy after installation.

Current software development thinking states a major change in this story, stating that users must be involved almost immediately when the project has a physical representation. This early user participation paradigm is motivated by the realization that it is far more economical and time-efficient to identify problems and improve the software early on than to deal with them after the project is finished.

This research explores how early user interaction is critical to the success of a project. It examines how software development methodologies have changed over time, stressing the shortcomings of the conventional covert strategy. We will explore the concrete advantages of involving users from project inception, drawing on case studies from real-world projects and scholarly studies. Early user involvement is shown to be critical to the success of software projects, including advantages such as reduced costs, enhanced communication, increased user happiness, and simpler development procedures.

This paper intends to assist stakeholders, developers, and project managers in comprehending the transformative power of early user involvement through case studies, analysis, and best practices. The secrets to opening the door to a more effective, user-aligned, and ultimately prosperous software development journey become apparent as we negotiate the complex interactions between developers and end users.


\subsection{Motivation}
Incorporating users into the early stages of the software development process is motivated by a multifaceted set of advantages. Firstly, the approach is centered on aligning the final product with the actual needs, preferences, and workflows of the end users. By involving users from the outset, developers gain valuable insights into user expectations, reducing the likelihood of developing a product that may not meet their requirements. This proactive measure for issue identification and resolution enables timely adjustments, minimizing the risk of costly rework and ensuring that the final product more accurately reflects user expectations. User satisfaction is another pivotal motivation. Involving users in the development process fosters a sense of ownership and engagement, as stakeholders witness their input being integrated into the evolving product. This collaborative approach contributes to higher levels of user satisfaction as the end product is more likely to align with their preferences and needs.

Cost savings and efficiency are integral components of the motivation behind early user involvement. Making adjustments early in the development process is generally more cost-effective than addressing issues post-implementation. This iterative approach allows for continuous refinement, contributing to a more streamlined and efficient development process. Additionally, the timely identification and resolution of potential challenges contribute to risk mitigation, ensuring that the project stays on track. Lastly, early user involvement is driven by a commitment to quality assurance. Beyond meeting functional requirements, this approach focuses on delivering a product with a superior user experience. It enables developers to consider factors such as usability, accessibility, and overall user satisfaction from the project's inception, reinforcing the emphasis on building software that genuinely serves its intended audience. In essence, the motivation behind early user involvement lies in creating a collaborative, responsive, and user-centric development process that yields successful outcomes.

\section{Methods \& Methodology}

\section{Results}

\section{Conclusion}

\section{References}
\begin{enumerate}
  \item \href{https://www.academia.edu/1012299/User_centered_design} {[1] User-Centered Design - Chadia Abras, Diane Maloney-Krichmar, Jenny Preece}
  \item \href{https://www.researchgate.net/publication/220208710_User_involvement_A_review_of_the_benefits_and_challenges} {[2] User involvement: a review of the benefits and challenges - SARI KUJALA}
  \item \href{https://www.researchgate.net/publication/348292073_The_impact_of_user_involvement_in_software_development_process} {[3] The impact of user involvement in software development process - Nouf Bin Saif, Mashael Almohawes, Nor Shahida Mohd Jamail }
  \item \href{https://chat.openai.com/ } {[4] Chat-GPT}

‌
‌

\end{enumerate}

\end{document}